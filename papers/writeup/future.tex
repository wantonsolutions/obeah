%%%%%%%%%%%%%%%%%%%%%%%%%%%%%%%%%%%%%%%%%%%%%
\section{Future work}
\label{sec:future}
%%%%%%%%%%%%%%%%%%%%%%%%%%%%%%%%%%%%%%%%%%%%%

Obeahs analysis technique is limited for the sake of simplicity, however there
are many additions which could be implemented to extend coverage, and the
replicability of faults.

\subsection{Function Learning}

When obeah finds that a particular control flow path is unsat, it blacklists
the path and considers it unreachable. This assumption is incorrect and made
for the sake of simplicity. Intermittented operations between braching statmes
could and likely manipulated the values of constraied variables and addtional
knowledge about the function could lead to more precise constrations. The other
side of the problem is that some proportion of control flow paths which obeah
belives are reachable may not be, or the determined assignment may be
incorrect. In such situations anonymous functions could be used in conjuction
with the control flow graph to better aproximate the behaivior of the program.
For example at each point at which a branch is appended to a trace the values
of network affected variables could be logged as well. Each edge of the CFG
could be interpreted as an anonymous function, where the input was the values
of each variable at the previous point, and the output is the value at the
current branch. When an unprobable path is generated Z3 could be passed the set
of branch conditions with the additional constraint that they must satisfy the
string of anonymous functions along the path. The research question is this
domain would be \emph{What inner circutry would produce efficient, and accurate
anonymous functions}

\subsection{Learing and retrying over multiple executions}

One limiation of obeah is that it requires a setup time to generate faults. The
faults are unpredicatable in that the paths which are executed are subject to
the same nondetermanism as the distributed system itself. This leads to the
problem of replicablilty. In order to determine preciesely the cause of
failures it would be useful to recreate failures over multiple runs, and gains
statistical confidence in the root of the failure. A possible exension for
obeah would be to write out an entire histroy of an execution which lead to a
failure and attempt to recreate it minimally over subsequent executions.
Similar work to this has been done in Demi~\cite{} demi uses delta debugging to
minimize the set messages needed to re create a failure in a distributed
system. Obeahs contribution to this work would be the means to generate the
fault in the first place.

\subsection{Monitering entire clusters}

Obeah currently only moniters insturmented nodes. This has the unfortunate side
effect that the actions of other nodes in the system are unobservable.
Nondtermanisim among other nodes may lead to failures that obeah did not
incite. False positive are a headache in verifiaction so facilities to minimize
them would increase the usefullness of the tool. Fully insturmenting a cluster,
and interleaving controled failures on all nodes would allow obeah to recreate
bugs faster, and output the entire sequece of events which lead to a failed
state.

\subsection{Solving paths across nodes}

Obeah only has the ability to reason about the control flow of a single node,
which limits it's power to affect other nodes in the system. A more complete
approach would be to construct a distributed control flow graph post exeuction
and attmpt to slove constraints on it. Post execution, in the case were all
nodes were instrumented, a control flow graph would exist for all nodes. Using
additional logging infomation such a vector clocks, sending and receiving
control flow paths could be linked. Using anonymous functions represed on other
nodes the payload of a message could be tailior to cause a failure which
resulted from one or more passed messages in the future. This approach would
lead to a refined approach for finding protocol specific bugs. Current
distributed verification techniques attempt to interleave all random
non-deterministic events, which makes no attempts to reason about tested
functionality. Intentionally triggering unprobably message sequences would
attack problems likely untested by developers.
