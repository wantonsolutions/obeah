\section{Execution}
\label{exection}

Our goal with Obeah is to manufacture byzantine faults during the execution of
a distributed system. To do so obeah permutes variables set by received message
to execute low probability, and unexpected paths through a program. In
Section~\ref{instrumentation} we discussed how obeah collects control flow
constraints from a programs source code, and instruments the source code to log
its control flow. In this section we describe how a programs execution is
profiled, how low probability paths are calculated, and the process by which
variables have their values perturbed.  Each instrumented node will exhibit
byzantine behaviour during it's execution. We leave it to system developers to
decide what degree of fault tolerance they wish to achieve. By running more
instrumented nodes during a test, the number of faults injected will increase.
For the purposes of discussion in this section, we will assume that an
executing system contains a single instrumented node, with the rest running
their original sources.

\subsection{Runtime Inputs, and Initialization}

During execution Obeah monitors execution and tabulates the control flow being
executed. Without any knowledge of the program being executed it would be
impossible to reason about which variables caused which paths to be executed,
and how to modify them to reach unusual states. As such, Obeah takes the control
flow data extracted during instrumentation as a runtime argument. One design
goal of obeah was to reduce developer effort in using the tool. Obeah does not
require that developers modify their code to take in control flow data as an
argument. Instead, each of Obeah's logging functions, such as \emph{Log} and
\emph{Taboo} check if obeah has been initialized when they are executed. Upon
initialization control flow data is read in from disk, using a pre-determined
name, from the processes local directory. Once initialized obeah will begin
profiling the system.

\subsection{Profiling}

Profiling an executing system is done by maintaining a runtime control flow
graph, and counting the number of times that one branch is taken after another.
The control flow graph is initialized as a set of unconnected points, each of
which is a conditional statement. A trace of conditionals executed is
maintained in the background. A new trace is started each time \emph{Taboo} is
executed. Executions of \emph{Taboo} are paired with received messages.
Therefore starting a trace from each \emph{Taboo} execution profiles how the
system behaves after a message is received. Each time an \emph{Log} function is
executed, the unique identifier of a conditional is appended to the trace. When
\emph{Taboo} is executed the current trace is processed, and a new one is
begun. Each trace has a head, which is a potently starting point on which to
compute a path. It is worthwhile to note that if the first conditional
encountered after receiving a message was a large switch statement, a common
occurrence in distributed systems, the number of path heads could be large. A
trace is processed by first adding any new heads to the control flow graph.
Second the trace is iterated over. The control flow graph is updated based on
the order of conditionals in the trace. Edges are added to the control flow
graph when two conditionals which had not been executed sequentially are
encountered in the trace. If an edge in the CFG existed then a count on the
edge is incremented. Over time the CFG will mature and obeah can calculate low
probability paths with high confidence.

\subsection{Finding Unlikely Paths}

Once a profile of a system has reach sufficient maturity, during our tests we
used a standard of 50 traces, Obeah will generate an unlikely path. To do so a
bounded breadth first search (BBFS) is performed on the CFG. All paths of a
specified depth $n$ are generated. The approximate probability of executing any
given path is calculated by multiplying the probabilities of each branch in the
path being taken. Post calculation the path with the lowest probability is
chosen to be the next path obeah will aim to execute. We note her that the
lowest likelihood path may not be feasible in the program. During path
calculation no attempt is made to reason about the paths semantics. Next we
discuss how low probability paths are reasoned about, and how variables are
perturbed to aim for that paths execution.

\subsection{Perturbation}
\label{runtime-pertubation}

Once a low likelihood path has been chosen, Obeah attempts to determine a
variable assignment which will cause the path to be executed. Obeah uses the Z3
constraint solver to reason about path constraints. An unlikely path is
formulated as the set of conditionals which need to be satisfied in order to
execute the path. Within each conditional also has a set of variables
associated with it. Obeah generates a program using Z3's python interface which
attempts to solve the path constraint. If the formula is unsat, then obeah
considers the path unfeasible. Unsat path constrains may well be executable
paths due to the internal logic of the program. But in order to remain light
weight, we do not attempt to reason about such cases. Instead the path is black
listed to prevent computation on it in the future. If a satisfying assignment
is found for the path constraints the satisfying variable values are set in the
running program.



