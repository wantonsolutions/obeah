\section{Instrumentation}
\label{sec:instrumentation}



Different flavors of testing exist for distributed systems; Typically the
tradeoff is between using either whitebox or blackbox methodologies. Black box
testing has the advantage of requiring no prior knowledge of the system, which
leads to more genral techniques. The tradeoff is that important information
about the system cannot be leveraged, thus blackbox methods test less
exaustively. In contrast whitebox methods use system specific information to
perform tests. Whitebox testing can perform more exaustive analysis and detect
deeper bugs, at the cost of generality and developer effort. 

Our goal is to use light weight whitebox techniques to perform system specific
testing, without incurring significant overhead, or requiring developer effort.
This section details our automatic insturmentation proceduring for learing a
programs control constraints, and injecting logging code for tracing a program
during runtime.

\subsection{Logging Points}

In order to profile the exection of a program at runtime we must log it's
control flow. Control flow can be profiled by logging a unique identifier after
each branch in a program. Obeah takes a program as input and analyzes it's AST.
At each brancing conditional, such as an if-else or switch statemet, code is
injected which logs which particular branch was executed. We use the position
of the conditional in the program as a unique identifier in the log statement.
Figure~\ref{insturmentation} shows the source code of a program before and
after instrumentation.

\subsection{Aggregating Conditionals}

Symbolic execution is useful for determining which variable values will lead to
a pariticular set of conditionals being executed. Unfortunatly symbolic
execution is expensive, and is infeasable to run on large systems while they
execute. We choose to only reason about the conditionals themselves to
mittigate the cost of symbolic execution. Each brach in a program has a set of
conditionals which must be satisfied in order for the brach to be taken.
Further, many conditionals are nested and require that many other branches are
not taken in order for them to be executed. Using static analysis on the AST of
a program we assign a set of conditional predicates to each branch of a
program. Collecting conditional predicates for a brach is done by backtracking
to a functions root from the branch statement. All conditionals which must be
satisfied for the branch to be executed are added to the collection. In
contrast all conditionals which must not be statisfied are negated and added to
the collection. For example in Listing~\ref{post-instrumentation} if the else
branch in the \emph{DoWork} case is to be executed two conditions must be
satisfied. \emph{msg.Size <= 100} and \emph{msg.Type == DoWork}. In this case
the collection of conditionals attached to log point \emph{4} would be
\emph{!(msg.Size > 100)} $\wedge$ \emph{msg.Type == DoWork}. Post
insturmentation all braches have an associated collection of constraints
associated with them. This information is persisted post instrumentation and
used during runtime to form larger constraints corresponding to paths through a
program.

\subsection{Variable Manipulation}

In order to influence a programs execution some of the variabes in a program
must be under the control of the tester. Our goal is to inject byzantine faults
which cause unlikely or unseen paths to be executed; As such we use the set of
variable set directly from receved messages as our controlable variables.
Currently variables which are set directly from an inbound message must be
specified by the devloper. These variable could be determined statically
through program sliceing, but that functionality is beyond the scope of this
project. The last point of instrumenation is the addition of
\emph{obeah.Taboo()} function calls. The \emph{Taboo} function takes pointers
to variables as arguemnts, along with identifiers for those variables.
\emph{Taboo} manipulates the values of variables passed in, so that particular
paths through the program will be likely be executed. We suggest that
\emph{Taboo} should be the first call made after a message is unmarshalled so
that variable manipuations will result in the same execution, as if a byzantine
message was received. The internals of \emph{Taboo} are discussed in
Section~\ref{execution-pertubation}.


\begin{lstlisting}
int main()
{
    msg := net.Read(buffer)
    
    switch msg.Type {
    case PING:
        net.Write("IIA")
    case DoWork:
        if msg.Size > 100 {
            spreadWorkAround(msg)
        } else {
            doTheWork()
        }
    }
}
\lable{pre-insturmeted}
\caption{source code of a simple program pre insturmentation}
\end{lstlisting}

\begin{lstlisting}
int main()
{
    msg := net.Read(buffer)
    obeah.Taboo(*msg)
    switch msg.Type {
    case PING:
        obeah.Log("1")
        net.Write("IIA")
    case DoWork:
        obeah.Log("2")
        if msg.Size > 100 {
            obeah.Log("3")
            spreadWorkAround(msg)
        } else {
            obeah.Log("4")
            doTheWork()
        }
    }
}
\lable{post-insturmeted}
\caption{source code of a simple program post insturmentation}
\end{lstlisting}


\begin{itemize}
    \item \textbf{Why instrument white vs blackbox}
    \item \textbf{what are we getting from instrumenting}
    \item \textbf{what tools are we using to instrument}
    \item \textbf{what is the scope of insturmentation}
    \item \textbf{what challenges are there in insturmenting}
    \item \textbf{what is the cost of instruementing}
\end{itemize}
