\section{Automatic Instrumentation}
\label{sec:instrumentation}



Different flavors of testing exist for distributed systems; Typically the
tradeoff is between using either whitebox or blackbox methodologies. Black box
testing has the advantage of requiring no prior knowledge of the system, which
leads to more general techniques. The tradeoff is that important information
about the system cannot be leveraged, thus blackbox methods test less
exhaustively. In contrast whitebox methods use system specific information to
perform tests. Whitebox testing can perform more exhaustive analysis and detect
deeper bugs, at the cost of generality and developer effort. 

Our goal is to use light weight whitebox techniques to perform system specific
testing, without incurring significant overhead, or requiring developer effort.
This section details our automatic instrumentation procedure for learning a
programs control constraints, and injecting logging code for tracing a program
during runtime.

\subsection{Logging Points}

To profile the execution of a program at runtime we must log it's control flow.
Control flow can be profiled by logging a unique identifier at each branch in a
program. Obeah takes a program as input and analyzes it's AST.  At each branch,
such as an if-else or switch statement, logging code is injected which logs
which particular branch was executed. We use the position of the conditional in
the program as a unique identifier in the log.
Listing~\ref{lst:pre-instrumentation}~\ref{lst:post-instrumentation} shows the
source code of a program before and after instrumentation.

\subsection{Aggregating Conditionals}

Symbolic execution is useful for determining which variable values will lead to
a particular set of conditionals being executed. Unfortunately symbolic
execution is expensive, and is unfeasible to run on large systems while they
execute. We choose to only reason about the conditionals themselves to
mitigate the cost of symbolic execution. Each branch in a program has a set of
conditionals which must be satisfied in order for the branch to be taken.
Further, many conditionals are nested and require that many other branches are
not taken in order for them to be executed. Using static analysis on the AST of
a program we assign a set of conditional predicates to each branch of a
program. Collecting conditional predicates for a branch is done by backtracking
to a functions root from the branch statement. All conditionals which must be
satisfied for the branch to be executed are added to the collection. In
contrast all conditionals which must not be satisfied are negated and added to
the collection. For example in Listing~\ref{lst:post-instrumentation} if the
else branch in the \emph{DoWork} case is to be executed three conditions must
be satisfied. \emph{msg.Type != PING}, \emph{msg.Size <= 100} and
\emph{msg.Type == DoWork}. In this case the collection of conditionals attached
to log point \emph{4} would be \emph{!(msg.Size > 100)} $\wedge$ \emph{msg.Type
== DoWork} $\wedge$ \emph{!(msg.Type == PING)}. Post insturmentation all
branches have a collection of constraints associated with them. This information
is persisted post instrumentation and used during runtime to form larger
constraints corresponding to paths through a program.

\subsection{Variable Manipulation}

To influence a programs execution some of the variables in a program must be
under the control of the tester. Our goal is to inject byzantine faults which
cause unlikely or unseen paths to be executed; As such we use the set of
variable set directly from received messages as our controllable variables.
Ideally variables which are set by obeah are directly extracted from inbound
messages. Detecting all such variables can be achieved through program
slicing~\cite{Ottenstein:1984}, but the technique is beyond the scope of this
project. Instead Obeah collects collects pointers to all in scope variables.
Trimming those variables down to those set by incoming messages is left to
developers. The last point of instrumentation is the addition of
\emph{obeah.Taboo()} function calls. The \emph{Taboo} function takes pointers
to variables as arguments, along with identifiers for those variables.
\emph{Taboo} manipulates the values of variables passed in, so that particular
paths through the program will be likely be executed. We suggest that
\emph{Taboo} should be the first call made after a message is unmarshalled so
that variable manipulations will result in the same execution, as if a byzantine
message was received. The internals of \emph{Taboo} are discussed in
Section~\ref{sec:execution-pertubation}.


\begin{lstlisting}[caption={source code pre-instrumentation},label={lst:pre-instrumentation}]
func main() {
    msg := net.Read(buffer)
    switch msg.Type {
    case PING:
        net.Write("IIA")
    case DoWork:
        if msg.Size > 100 {
            spreadWorkAround(msg)
        } else {
            doTheWork()
        }
    }
}
\end{lstlisting}

\begin{lstlisting}[label={lst:post-instrumentation}, caption={Source code post instrumentation}]
func main() {
    msg := net.Read(buffer)
    obeah.Taboo(*msg)
    switch msg.Type {
    case PING:
        obeah.Log("1")
        net.Write("IIA")
    case DoWork:
        obeah.Log("2")
        if msg.Size > 100 {
            obeah.Log("3")
            spreadWorkAround(msg)
        } else {
            obeah.Log("4")
            doTheWork()
        }
    }
}
\end{lstlisting}

The result of instrumenting a program with obeah is a program with instrumented
source code, and a profile of the programs control flow. The following section
describes how Obeah pertubates instrumented programs at runtime, and how
control flow information is used to influence Obeah's perturbations.

